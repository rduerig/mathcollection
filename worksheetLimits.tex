\documentclass[a4paper, 12pt, ngerman]{exam}
\usepackage{preamble/draft}
\usepackage{pgfplots}
\pgfplotsset{compat = newest}

% layout for listings
\lstset{
 morekeywords={for, in, if, then, endfor},
 basicstyle=\ttfamily,
 keywordstyle=\bfseries,
 frame=single,
 frameround=tttt,
 escapeinside={|}{|},
 numbers=left,
 numberstyle=\tiny,
 breaklines=true
}

% page layout
\geometry{a4paper,left=2.5cm,right=2.5cm, top=3cm, bottom=3cm}

% font layout
\titleformat{\chapter}[display]
  {\normalfont\sffamily\huge\bfseries}
    {\chaptertitlename\ \thechapter}{20pt}{\Huge}
%\titleformat*{\chapter}{\LARGE\bfseries\sffamily}
\titleformat*{\section}{\Large\bfseries\sffamily}
\titleformat*{\subsection}{\large\bfseries\sffamily}
\titleformat*{\subsubsection}{\normalsize\bfseries\sffamily}
\titleformat*{\paragraph}{\small\bfseries\sffamily}
\titleformat*{\subparagraph}{\footnotesize\bfseries\sffamily}



\newcommand{\examtype}{Übung Grenzwerte}
\newcommand{\examno}{3}
\newcommand{\examdate}{26.04.2022}
\newcommand{\subject}{Mathematik}
\newcommand{\examclass}{}

%\addpoints
\pointpoints{Punkt}{Punkte}
\bonuspointpoints{Bonuspunkt}{Bonuspunkte}
\renewcommand{\solutiontitle}{\noindent\textbf{Lösung:}%
\enspace}

\chqword{Frage}
\chpgword{Seite}
\chpword{Punkte}
\chbpword{Bonuspunkte}
\chsword{Erreicht}
\chtword{Gesamt}

\hpword{Punkte:} % Punktetabelle
\hsword{Ergebnis:}
\hqword{Aufgabe:}
\htword{Summe:}

\pagestyle{headandfoot}
\firstpageheadrule
\runningheadrule
\firstpageheader{\examclass}{\large{\textbf{\subject}}\\ \large{\examtype\ \examno}}{\examdate}
\runningheader{\examclass}{\large{\textbf{\subject}}\\ \large{\examtype\ \examno}}{\examdate}
\firstpagefooter{}{Seite \thepage\ von \numpages}{}
\runningfooter{}{Seite \thepage\ von \numpages}{}

\qformat{\textbf{Aufgabe \thequestion} \hfill}
\pointformat{}

\pointsinrightmargin

\renewcommand{\familydefault}{\sfdefault}

\colorgrids
\definecolor{GridColor}{gray}{0.7}

\tikzumlset{fill class=white}


%\printanswers

\begin{document}

\begin{questions}
 
  \question
  Gegeben Sei die folgende Funktion: \[ f(x) = \frac{2}{x+1} \]
  \begin{parts}
    \part
    Bestimmen Sie den Definitionsbereich $D$ der Funktion $f(x)$.
    \begin{solution}
      \[ D = \left\{ x \in \mathbb{R}:\ x+1 \neq 0 \right\} \]
      bzw. \[ D = \left\{ x \in \mathbb{R}:\ x \neq -1 \right\} \]
    \end{solution}
    \part
    Untersuchen Sie das Verhalten der Funktion $f(x)$ für $x \to -1$.
    \begin{solution}
      \[ \lim \limits_{\substack{x \to -1\\ x > -1}} f(x) = \infty \]
      und \[ \lim \limits_{\substack{x \to -1\\ x < -1}} f(x) = -\infty \]
    \end{solution}
    \part
    Untersuchen Sie das Verhalten der Funktion $f(x)$ im Unendlichen, also für $x \to \infty$.
    \begin{solution}
      \[ 
        \begin{aligned}
          \lim \limits_{x \to \infty} f(x) \\
          = \lim \limits_{x \to \infty} \cfrac{x\cfrac{2}{x}}{x(1+\frac{1}{x})} \\
          = \lim \limits_{x \to \infty} \cfrac{\cfrac{2}{x}}{1+\frac{1}{x}} \\
          = \cfrac{0}{1+0} \\
          = 0
        \end{aligned}
      \]
    \end{solution}
  \end{parts}

  \question
  Gegeben Sei die folgende Funktion: \[ f(x) = \frac{4x+1}{x} \]
  \begin{parts}
    \part
    Bestimmen Sie den Definitionsbereich $D$ der Funktion $f(x)$.
    \begin{solution}
      \[ D = \left\{ x \in \mathbb{R}:\ x \neq 0 \right\} \]
    \end{solution}
    \part
    Untersuchen Sie das Verhalten der Funktion $f(x)$ im Unendlichen, also für $x \to \infty$.
    \begin{solution}
      \[ 
        \begin{aligned}
          \lim \limits_{x \to \infty} f(x) \\
          = \lim \limits_{x \to \infty} (\frac{4x}{x} + \frac{1}{x}) \\
          = \lim \limits_{x \to \infty} (4 + \frac{1}{x}) \\
          = 4 + 0 \\
          = 4
        \end{aligned}
      \]
    \end{solution}
  \end{parts}

  \question
  Gegeben Sei die folgende Funktion: \[ f(x) = \frac{2x^2-x+3}{x^2+3x} \]
  \begin{parts}
    \part
    Bestimmen Sie den Definitionsbereich $D$ der Funktion $f(x)$.
    \begin{solution}
      \[
        \begin{aligned}
          x^2 + 3x = 0 \\
          \rightarrow x_1 = -\frac{3}{2} + \sqrt{\frac{9}{4}} = -\frac{3}{2} + \frac{3}{2} = 0 \\
          \rightarrow x_2 = -\frac{3}{2} - \frac{3}{2} = -3
        \end{aligned}
      \]
      Damit ist:
      \[ D = \left\{ x \in \mathbb{R}:\ x \neq 0 \land x \neq -3 \right\} \]
    \end{solution}
    \part
    Untersuchen Sie das Verhalten der Funktion $f(x)$ im Unendlichen, also für $x \to \infty$.
    \begin{solution}
      \[ 
        \begin{aligned}
          \lim \limits_{x \to \infty} f(x) \\
          = \lim \limits_{x \to \infty} \frac{x^2(2-\frac{1}{x}+\frac{3}{x^2})}{x^2(1+\frac{3}{x})} \\
          = \lim \limits_{x \to \infty} \frac{2-\frac{1}{x}+\frac{3}{x^2}}{1+\frac{3}{x}} \\
          = \cfrac{2-0+0}{1+0} \\
          = 2
        \end{aligned}
      \]
    \end{solution}
  \end{parts}

\end{questions}

\end{document}
